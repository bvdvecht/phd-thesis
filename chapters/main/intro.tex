\chapter{Introduction}
\label{chp:intro}

This work shows a history of projects that build upon each other.
The work has been part of a larger ecosystem in which multiple projects worked towards the goal of executing quantum network applications.
Many software packages have been written.

The original goal of this PhD project was to research compilation of quantum network programs.
It turned out that first many other things needed to be fleshed out.
This thesis can be seen as following the road towards fleshing this out, culminating in a chapter with ideas on compilation.

\section{A brief history of (quantum) computers}
Computers used to be people.
Later: mechanized. Manipulating 0s and 1s.
Quantum mechanics introduced new kind of fundamental value.
Applications/use-cases.
Internet started out as...
Quantum communication history.
Development of tools: languages, compilers, simulators, online access (IBM).
State of the art for quantum networks in 2020: CQC and simulaqron.

This thesis adds the following...




Hardware alone is not enough to realize applications.
Software and abstractions are needed for scalability.
This thesis provides abstractions and in general tries to provide structure to the problems of application realization,
by introducing new concepts, paradigms, and considerations.


This thesis presents a lot of design.
All chapters follow the order design-considerations, architecture, evaluation.

\section{Overview of chapters}
Quantum internet application frameworks have been evolving for some time.
Around 2020, when this PhD project started, CQC existed and SimulaQron.
No hardware implementation.

\lipsum[40-50]
A sentence with a reference~\cite{dahlberg_2022_netqasm}.


\section{Data and Software Availability}
\lipsum[51]


\section{Thesis Outline}
\lipsum[52]


\begin{xstretch}
\printbibliography[heading=subbibintoc,title={References},notcategory=noprint]
\end{xstretch}
