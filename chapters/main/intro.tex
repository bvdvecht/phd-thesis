\chapter{Introduction}
\label{chp:intro}

This work shows a history of projects that build upon each other.
The work has been part of a larger ecosystem in which multiple projects worked towards the goal of executing quantum network applications.
Many software packages have been written.

The original goal of this PhD project was to research compilation of quantum network programs.
It turned out that first many other things needed to be fleshed out.
This thesis can be seen as following the road towards fleshing this out, culminating in a chapter with ideas on compilation.

\section{A brief history of (quantum) computers}

\todo{Update below generated text}
Computers, in the early days, were not the machines we know today but rather human beings.
People, referred to as "computers," manually performed complex calculations.
As time progressed, mechanical devices took over these tasks, and with the advent of modern technology, machines capable of manipulating binary code—0s and 1s—became the norm.

The field of computing underwent another revolution with the introduction of quantum mechanics, which brought a new kind of fundamental value to the process.
Quantum computing, unlike classical computing, relies on principles such as superposition and entanglement, allowing it to handle complex problems far more efficiently.

The rise of quantum computing opened up a wealth of potential applications and use cases.
At the same time, the Internet, which initially began as a small network for information exchange among scientists, expanded into a global communications infrastructure.
This expansion also paved the way for the emergence of quantum communication, an area rooted in secure information exchange via quantum principles.

As the field of quantum computing advanced, tools were developed to support its growth.
New programming languages, compilers, simulators, and even online platforms—such as IBM's quantum computing access—enabled researchers and developers to explore quantum algorithms and systems.

By 2020, the state of the art in quantum networks had reached significant milestones.
Projects like CQC (Cambridge Quantum Computing) and tools like Simulaqron provided researchers with environments to simulate and experiment with quantum networks, further advancing the practical possibilities of quantum technology.



Hardware by itself is insufficient to fully realize practical applications, especially in complex and scalable systems.
To achieve true scalability and functionality, software and higher-level abstractions are essential.
These abstractions create the necessary frameworks that allow hardware to be used effectively across different scenarios and at larger scales.

This thesis addresses the need for such abstractions, offering structure to the challenges associated with realizing applications.
It introduces new concepts, paradigms, and considerations designed to guide the development of scalable solutions, providing a structured approach to solving the problems that arise in application realization.

\todo{end generated text}


This thesis presents a lot of design.
All chapters follow the order design-considerations, architecture, evaluation.


\section{Terminology}
\begin{itemize}
  \item Application
  \item Program
  \item EPR pair
  \item Quantum circuit or circuit
  \item Qubit
  \item Entanglement
\end{itemize}

\section{Overview of chapters}
Quantum internet application frameworks have been evolving for some time.
Around 2020, when this PhD project started, CQC existed and SimulaQron.
No hardware implementation.


\section{Data and Software Availability}
\lipsum[51]


\section{Thesis Outline}
\lipsum[52]


\begin{xstretch}
\printbibliography[heading=subbibintoc,title={References},notcategory=noprint]
\end{xstretch}
