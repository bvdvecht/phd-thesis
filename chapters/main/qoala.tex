\chapter
 [Qoala: an Application Execution Environment for Quantum Internet Nodes]
 {Qoala: an Application Execution Environment for Quantum Internet Nodes}
\label{chp:qoala}

\begin{abstract}
In \cref{chp:qnodeos}, we presented a first-of-its-kind operating system for programmable quantum network nodes, called QNodeOS.
In this chapter, we present an extension of QNodeOS called Qoala, which introduces
(1) a unified program format for hybrid interactive classical-quantum programs, providing a well-defined target for compilers, and 
(2) a runtime representation of a program that allows joint scheduling of the hybrid classical-quantum program, multitasking, and asynchronous program execution.
Based on concrete design considerations, we put forward the architecture of Qoala, including the program structure and execution mechanism.
We implement Qoala in the form of a modular and extendible simulator that is validated against real-world quantum network hardware (available online).
However, Qoala is not meant to be purely a simulator, and implementation is planned on real hardware.
We evaluate Qoala's effectiveness and performance sensitivity to latencies and network schedules using an extensive simulation study.
Qoala provides a framework that opens the door for future computer science research into quantum network applications, including scheduling algorithms and compilation strategies that can now readily be explored using the framework and tools provided. 
\end{abstract}

\blfootnote{
This chapter is based on the preprint:
\fullcite{qoala_noprint}.
}

\iffullchapters
\section{Introduction}
\label{sec:introduction}

\begin{figure}% [ht]
    \centering
    \includegraphics[scale=0.35]{figures/qoala/program_illustration.png}
    \caption{Example application consisting of two hybrid classical-quantum programs (on Nodes 1 and 2) including
        (1) Entanglement generation between two qubits (circles) in a synchronized time slot (defined by  network controller).
        (2) A local measurement of qubit A at Node 1 resulting in a classical outcome bit (destroying the qubit)
        (4) Communication of the classical bit from Node 1 to Node 2 (taking non-deterministic time)
        (5) Execution of a quantum circuit on qubit B at Node 2 depending on the classical bit. The quality of qubit B has degraded during the time elapsed since (1). 
        (6) Node 2 measures qubit B and outputs the classical result.
    }
    \label{fig:program_illustration}
\end{figure}

Advances in quantum computing and quantum communication technologies are paving the way for a \textit{quantum internet}~\cite{wehner2018quantum, kimble2008quantum}, where quantum applications are executed across multiple network nodes.
Examples of such applications include quantum key distribution (QKD)~\cite{bennett2014quantum, ekert1991quantum} and blind quantum computation (BQC) \cite{broadbent2009universal, arrighi2006blind} from a client to a quantum cloud server.
A multi-node quantum internet application is partitioned into separate single-node \textit{programs} (e.g. a client program and a server program in BQC) that run concurrently on different network nodes. To support security sensitive applications, each program performs local classical and quantum computations on its own private node, and programs interact with each other only via classical message passing and entanglement generation. This is in sharp contrast to distributed quantum computing (see e.g.~\cite{cacciapuoti2019quantum}), where all nodes can be accessed and controlled by a single program. 

The single-node programs that constitute a quantum internet application are hybrid in nature (see Fig~\ref{fig:program_illustration}):
First, they contain quantum operations, such as local quantum gates and measurements (e.g. to perform a server computation in BQC), and entanglement generation (e.g. to produce key in QKD). Entanglement is a special property of two quantum bits (qubits) that forms a key resource for quantum internet applications. 
All quantum operations are executed on quantum processors that can store, manipulate and measure quantum information, where small networks including such processors have been realized using different quantum hardware platforms including, for example,  Nitrogen-Vacancy (NV) centers in diamond~\cite{pompili2021realization}, and Ion Traps~\cite{krutyanskiy2023entanglement}.
Second, programs need to perform classical operations, such as message passing (e.g., a BQC client program sending desired measurement bases to the BQC server), and local classical processing (e.g., post-processing measurement outcomes in QKD).

Realizing the execution of quantum internet applications presents unique challenges (see Section~\ref{sec:design_considerations}): 
First, a program for a quantum internet application is not merely a hybrid of classical and quantum code segments; these segments are also highly \textit{interactive}: classical and quantum code may run concurrently, communicating and influencing each other.
E.g., a quantum circuit may "pause" halfway, keeping quantum states in memory, and wait for a value from a classical segment (e.g. a classical message from a remote node) before continuing.
Quantum memories have limited lifetimes, meaning qubits are subject to decoherence, degrading their quality over time. This introduces the need 
control the joint schedule of the classical and quantum segments of the program to reach desired levels of application performance.

Second, a compiler should be able to optimize the whole program including both classical and quantum code, as well as to provide information that can be used in our architecture to align and inform scheduling decisions. 

Finally, we are faced with a mix of time scales:
on the one hand, entanglement generation requires a very precise network schedule that is agreed ahead of time between the network nodes~\cite{dahlberg2019link}. On the other hand, classical messages are exchanged asynchronously between the nodes without guaranteed message delivery times. This motivates an architecture in which different segments of the system may operate at different levels of timing precision. 

\subsection{Main contributions}
We propose the first architecture, Qoala, that addresses these challenges. Qoala is an execution environment tailored to programmable quantum internet nodes, accommodating the \textbf{hybrid, interactive, networked, and asynchronous nature} of quantum internet applications. 

\textbf{Unified program format for hybrid-classical quantum programs:}
Qoala defines a unified program format for executables, encompassing classical and quantum (networked and local) code, and defining basic blocks.
This paves the way for a joint optimization of the classical and quantum code by a compiler.
% The program format is also made such that it can be the output of a compiler, making use of basic blocks.

\textbf{Runtime representation allowing scheduling:} Qoala separates the static unified program format from a runtime representation consisting of \textit{tasks}. 
This paves the way to design and implement algorithms for scheduling the quantum program in order to meet deadlines imposed by decoherence of the quantum memory.  
To provide advice to the scheduler on deadlines to achieve a desired program performance, programs can specify advice for timing and prioritization depending 
on the quantum hardware capabilities of the node. 
The separation of a static program from its runtime tasks also allows for the programmer to define asynchronous code segments, the execution of which is decided by the scheduler alone.
This is the first architecture that allows for effective scheduling control of hybrid interactive classical-quantum programs, thus addressing a critical issues in the successful execution of quantum internet applications.

\textbf{Integration with quantum network stack:}
Qoala integrates with an existing quantum network stack~\cite{dahlberg2019link} implemented on NV centers in diamond~\cite{pompili2022experimental} for realizing entanglement generation between nodes. This opens the door for Qoala to be implemented on such networks.

\textbf{Implementation in hardware validated simulation:} We implement the proposed architecture as a \textbf{modular and composable simulator}, which enables the evaluation of different execution strategies and techniques.
The simulation is validated against real-world quantum hardware implementations, opening the door to understand performance tradeoffs and requirements for Qoala's implementation. Specifically, the simulator allows 
configuring different hardware parameters, latencies, and software component organizations, to evaluate implementation choices of Qoala in simulation. 

Using the implementation we demonstrate the effectiveness and feasibility of our proposed architecture on different types of quantum hardware, including its ability to schedule and multitask applications using a number of existing scheduling methods (EDF, FCFS).
We continue to examine tradeoffs in the classical and quantum performance metrics of using different types of scheduling approaches. 
We examine Qoala's improvement over NetQASM~\cite{dahlberg2022netqasm} in enabling hybrid classical-quantum compilation possibilities. 
Finally, we study trends in application performance when varying the amount of concurrency, and examine the impact of a network schedule for entanglement generation on the performance of Qoala.

We highlight the role of Qoala in opening the door for computer science research. We make our simulator available as open source~\cite{qoala2023simulator}, paving the way for computer scientists to conduct further research, e.g., into the design of compilers, or schedulers that can readily be tested using the simulator. 

The remainder of this paper is structured as follows. Section~\ref{sec:related_work} compares our work to related studies. In Section~\ref{sec:design_considerations} we explain important context and terminology, followed by considerations that we used to design our architecture (Section~\ref{sec:architecture}). Section~\ref{sec:implementation} discusses our implementation and Section~\ref{sec:evaluation} provides evaluation results using this implementation. We conclude and give suggestions on future work (Section~\ref{sec:conclusion}).


\input{chapters/main/qoala/related_work}
\input{chapters/main/qoala/design_considerations}
\section{Compiler Architecture}

\subsection{Overview}
The figure below presents the general architecture of the Qoala compiler:

\begin{figure*}[ht]
    \centering
    \includegraphics[scale=1.0]{figures/compiler/compiler-arch.png}
    \caption{Qoala compiler stages}
    \label{fig:qoala_compiler_stages}
\end{figure*}


The core steps in the Qoala compilation process are:

\begin{itemize}
\item Programmer writes source code in Python using the Qoala Python SDK.
\item The Python SDK itself provides a `compile` function that executes the source code
  which produces a `.mlir` file containing the program code in QoalaHIR format.
\item This Python SDK implements \textit{some} functionality present in the "frontend" of a classical
  compiler, for example, a semantical analysis to check the validity of the classical and
  quantum operations specified in the program 
\item A separate `opt`-like tool (built using LLVM/MLIR) takes the `.mlir` as input and produces
  (after several applying passes) a `.iqoala` file.
\item The `.iqoala` file can then be executed by a Qoala runtime.
\end{itemize}


\subsection{High-level choices}
\begin{itemize}
\item Why python
  Since it first launch, python was positioned as a programming language that allows the programmer to quickly develop
  solutions with less lines of code. Lowering the usual barriers from other programming language is what allowed python
  to gain popularity among the scientific community.
  The Qoala platform goes in this line by allowing the scientific community to write quantum internet programs in a
  language that is easy to use, that is easy to understand, and that is known by the scientific community.
\item Why MLIR
  MLIR is a modular and extensive compiler infrastructure that allows extending the LLVM infrastructure, easing the
  developing of domain-specific representations and allowing reusing other pieces of the LLVM compiler infrastructure.
  By allowing the definiton of \textit{dialects}, MLIR allows extending the intermediate representations of the clang compiler
  pipeline, enabling the conversion of programs between different dialects, and also alowing the re-usage of the
  optimizations and analysis implemented in other dialects.
  All these feature make MLIR a perfect candidate for creating the compiiler for hybrid quantum internet programs.
\item Why iqoala as output
\item Why protocol description
\end{itemize}
    

\subsection{Python SDK}
The Python SDK is a library containing classes and methods that allow:
- expressing quantum internet program logic, and
- producing a QoalaHIR representation of the source code

The general should look like this:

\begin{pycode}
class MyProgram(QoalaProgram):
    def main(self, ctx: QoalaContext):
        q = ctx.new_qubit()
        e = ctx.entangle_keep("Bob")
        ctx.cnot(q, e)
        m = ctx.measure(q)
        ctx.add_return(m)

program = MyProgram()
mlir = program.compile()  # `compile` is defined in `QoalaProgram`
with open("program.mlir", "w") as f:
    f.write(mlir.text())
\end{pycode}

\todo{Corresponding MLIR code per IR}

The Python SDK acts as the "frontend" of the qoala compiler. In this sense,
the output of the Python SDK is a `.mlir` file in the QoalaHIR dialect.
By using Python as a the high-level language of the programs, we can take advantage
of the python interpreter to perform the parsing of the code and an initial
syntactical analysis.
To complete the frontend analysis, the Qoala libraries could also embed a
semantical analysis to check the correctness of the applied quantum operations.
As mentioned before, the main goal of this Python SDK (frontend) will be to
create a High Level representation of the program, using the QoalaHLIR
language, which will be the main input of the optimization pipeline int he next stages.


\subsection{MLIR-based optimizer pipeline}
The Qoala compiler also defines an optimization pipeline that analyses and
transforms the code to improve its performance and prepare it for the final
translation in the iQoala format.

To this end, the optimization pipeline makes use of the MLIR framework to
define intermediate representations as *dialects*. Later, the optimization
tools can apply *passes* on the intances of the program expressed using the
MLIR dialects.


\subsection{Representations and dialects}
The Qoala compiler uses three intermediate representations (IRs) when in the "MLIR-phase".
These are the High-level Intermediate Representation (HIR), Mid-level Intermediate Representation (MIR), and the Low-level Intermediate Representation (LIR).
Each IR is associated with a particular set of MLIR dialects.
The Qoala compiler uses four custom MLIR dialects, called `QNet`, `QMem`, `QoalaHost`, and `Netqasm`, explained in more detail below.

\begin{itemize}
\item HIR (High-level): A higher-level IR, where operations are closely related
  to the python source.
  Quantum operations are represented using the `QNet` dialect, which consume and produce quantum *values*.
  Programs in the HIR format use the following dialects: `QNet`, `arith`, `scf`, `affine`, `async`, `tensor`.
\item MIR (Mid-level): A mid level IR, similar to HIR but with explicit memory locations, using the `QMem` dialect instead of `QNet`.
\item LIR (Low-level): A lower level IR, where operations are closer to the
  classical and quantum assembly instructions.
  Quantum operations are represented using the `Netqasm` dialect, which take quantum *registers* ("quantum memory pointers") as operands
  and have side-effects on the quantum value stored in the registers.
  Programs in the LIR format use the following dialects: `QoalaHost`, `Netqasm`, `arith`, `scf`, `affine`, `async`.
\end{itemize}

Tranforming from HIR to LIR hence involves lowering QNet operations to QoalaHost and Netqasm operations.
On top of that, both HIR and LIR have additional constraints about the structure of code in that representation which need to be respected.



\begin{itemize}
\item High-level MLIR Dialect (QoalaHIR)
    \begin{itemize}
        \item Hybrid classical-quantum
        \item No memory allocation
        \item Timing and fidelity annotations
        \item Entanglement as abstract operation
    \end{itemize}
\item Low-level MLIR Dialect (QoalaLIR)
    \begin{itemize}
        \item Explicit split between classical and quantum code
        \item Memory allocation
        \item Timing and fidelity annotations
        \item Entanglement as abstract operation
    \end{itemize}
\item Translation passes:
    \begin{itemize}
        \item HIR -> LIR
        \item Codegen LIR -> QoalaHost instructions
        \item Codegen LIR -> NetQASM
    \end{itemize}
\item Optimization passes:
    \begin{itemize}
        \item Memory mapping on quantum LIR
        \item Find best split between cLIR and qLIR when translating HIR -> LIR (can be pass on HIR before translation)
        \item Convert fidelity constraints into timing constraints
        \item Extract constraints from protocol description into HIR annotations
    \end{itemize}
\end{itemize}


\subsubsection{General optimization pipeline}
The MLIR optimization pipeline applies the folliwng operations:
\begin{itemize}
\item apply optimzation passes on the QoalaHIR code
\item transform QoalaHIR to QoalaLIR
\item apply optimzation passes on the QoalaLIR code
\item generate `.iqoala` code from QoalaLIR
\end{itemize}

This optimization pipeline introduce modifications in different levels of the process:
\begin{itemize}
\item At HIR level: operations can be re-order, respecting the restrictions specified
  [in the HIR specification document](dialects/HIR.md).
\item Conversion from HIT to LIR: High-level qubit allocations instructions are exploded
  into the lower-level qubit allocation operations, which includes allocation,
  initialization and value writing/reading.
\item At LIR level: operations belonging to the same local quantum functions are grouped
  into functions ("functionize") to aid the scheduling at runtime.
\end{itemize}


\subsection{Optimizations}
We consider the following optimizations:
\begin{itemize}
\item qubit mapping optimization (already exists)
\item gate-level optimization (already exists)
\item classical-quantum division (limit host-qnodeos communication and limit qubit lifetimes)
\item inserting deadlines to try and meeting timing and fidelity constraints
\end{itemize}


\subsubsection{Memory Alloc (i.e. Qbits allocation)}
- Entanglement move to memory qubti??
- Use/keep of subroutines in iqoala ??

Idea: try to stitch together separate functions/subroutines to get one circuit and apply standard circuit optimizations on it.
Does not really work: in general programs can have many different circuits at runtime depending on control-flow.

Maybe: first functionize, then check qptrs that are passed into/out of function calls? Then try to map these pointers to as few qubit IDs as possible.

??? Legalize to specific Unit Module. E.g. for NV with 2 qubits, a cphase gate must be between a comm qubit and a mem qubit. So, values used in cphase operation must come from comm pointer and mem pointer respectively. 
- Do a normal pass? or
- Bake constraint into dialect? (but then: dialect needed for each different UM)

\subsubsection{Pass: memory alloc by assigning constants?}
Maybe first use QubitPointers. Then with an optional pass: assign constant values to these pointers. (If pass not used, then Host must provide qubit IDs as params to each NetQASM routine)



\subsection{Protocol description}
\begin{itemize}
    \item Fidelity constraints in protocol. Needed since otherwise protocol may not be valid anymore (e.g. BQC is not secure anymore if EPRs are below certain fidelity). Indirectly informs (through compilation and then capability negotiation) the demand registration.
    \item Timing constraints in high-level program code. Why needed?
\end{itemize}

\section{Implementation}
\todo{IMprove this section}

We report on a preliminary implementation of our compiler design, using MLIR and LLVM.
This implementation has focused mainly on translating high-level code to a Qoala executable, using Intermediate Representations (IRs) as proposed in the previous section.
We have not yet implemented any optimizations, but note that our implementation can easily incorporate these by writing these as \emph{passes} (a construct in LLVM) that are applied on the IRs.

\subsection{IRs}
Our compiler is implemented using the LLVM framework (C++), making use of the MLIR subproject of MLIR.
The Qoala compiler uses three intermediate representations (IRs).
These are the High-level Intermediate Representation (HIR), Mid-level Intermediate Representation (MIR), and the Low-level Intermediate Representation (LIR).
Each IR is associated with a particular set of MLIR dialects.
The Qoala compiler uses four custom MLIR dialects, called \texttt{QNet}, \texttt{QMem}, \texttt{QoalaHost}, and \texttt{Netqasm}, explained in more detail below.

\begin{itemize}
\item HIR (High-level): A higher-level IR, where operations are closely related
  to the python source.
  Quantum operations are represented using the \texttt{QNet} dialect, which consume and produce quantum *values*.
  Programs in the HIR format use the following dialects: \texttt{QNet}, \texttt{arith}, \texttt{scf}, \texttt{affine}, \texttt{async}, \texttt{tensor}.
\item MIR (Mid-level): A mid level IR, similar to HIR but with explicit memory locations, using the \texttt{QMem} dialect instead of \texttt{QNet}.
\item LIR (Low-level): A lower level IR, where operations are closer to the
  classical and quantum assembly instructions.
  Quantum operations are represented using the \texttt{Netqasm} dialect, which take quantum *registers* ("quantum memory pointers") as operands
  and have side-effects on the quantum value stored in the registers.
  Programs in the LIR format use the following dialects: \texttt{QoalaHost}, \texttt{Netqasm}, \texttt{arith}, \texttt{scf}, \texttt{affine}, \texttt{async}.
\end{itemize}

\subsection{Passes}
The MLIR optimization pipeline applies the folliwng operations:
\begin{itemize}
\item apply optimzation passes on the QoalaHIR code
\item transform QoalaHIR to QoalaLIR
\item apply optimzation passes on the QoalaLIR code
\item generate `.iqoala` code from QoalaLIR
\end{itemize}

This optimization pipeline introduce modifications in different levels of the process:
\begin{itemize}
\item At HIR level: operations can be re-order, respecting the restrictions specified
  [in the HIR specification document](dialects/HIR.md).
\item Conversion from HIR to LIR: High-level qubit allocations instructions are exploded
  into the lower-level qubit allocation operations, which includes allocation,
  initialization and value writing/reading.
\item At LIR level: operations belonging to the same local quantum functions are grouped
  into functions ("functionize") to aid the scheduling at runtime.
\end{itemize}


The Qoala compiler also defines an optimization pipeline that analyses and
transforms the code to improve its performance and prepare it for the final
translation in the iQoala format.

To this end, the optimization pipeline makes use of the MLIR framework to
define intermediate representations as *dialects*. Later, the optimization
tools can apply *passes* on the intances of the program expressed using the
MLIR dialects.


\subsection{Lowering}
Transforming from HIR to LIR hence involves lowering QNet operations to QoalaHost and Netqasm operations.
On top of that, both HIR and LIR have additional constraints about the structure of code in that representation which need to be respected.

\subsection{Python SDK}
We have implemented a new, bare-bones, SDK for the purpose of trying our compiler; in the future we aim to merge this with the NetQASM SDK such that programs written in this existing SDK can immediately be compiled using our compiler.

\begin{itemize}
\item Programmer writes source code in Python using the Qoala Python SDK, similar to the NetQASM SDK (\cref{chp:netqasm}).
\item The Python SDK itself provides a \texttt{compile} function that executes the source code
  which produces a \texttt{.mlir} file containing the program code in QoalaHIR format.
\item This Python SDK implements \textit{some} functionality present in the "frontend" of a classical
  compiler, for example, a semantical analysis to check the validity of the classical and
  quantum operations specified in the program 
\item A separate \texttt{opt}-like tool (built using LLVM/MLIR) takes the \texttt{.mlir} as input and produces
  (after several applying passes) a \texttt{.iqoala} file.
\item The \texttt{.iqoala} file can then be executed by a Qoala runtime.
\end{itemize}

The Python SDK acts as the "frontend" of the qoala compiler. In this sense,
the output of the Python SDK is a \texttt{.mlir} file in the QoalaHIR dialect.
By using Python as a the high-level language of the programs, we can take advantage
of the python interpreter to perform the parsing of the code and an initial
syntactical analysis.
To complete the frontend analysis, the Qoala libraries could also embed a
semantical analysis to check the correctness of the applied quantum operations.
As mentioned before, the main goal of this Python SDK (frontend) will be to
create a High Level representation of the program, using the QoalaHLIR
language, which will be the main input of the optimization pipeline int he next stages.




\subsection{Example lowering}

\begin{pycode}
class MyProgram(QoalaProgram):
    def main(self, ctx: QoalaContext):
        q = ctx.new_qubit()
        e = ctx.entangle_keep("Bob")
        ctx.cnot(q, e)
        m = ctx.measure(q)
        ctx.add_return(m)

program = MyProgram()
mlir = program.compile()  # `compile` is defined in `QoalaProgram`
with open("program.mlir", "w") as f:
    f.write(mlir.text())
\end{pycode}

\paragraph{HIR}
\begin{pycode}
qnet.func @main() -> i1 {
    %q = qnet.eprs {N = 1, remote = @Bob} : !qnet.qubit
    %q2 = qnet.hadamard %q : !qnet.qubit
    %floats1 = qnet.recv_floats {remote = @Bob, length = 1 : i32} : tensor<1xf32>
    %t1 = tensor.extract %floats1[%zero] : tensor<1xf32>
    %q3 = qnet.rot_x %q2, %t1 : !qnet.qubit
    %m = qnet.measure %q3 : i1
    qnet.return %m : i1
}
\end{pycode}

\paragraph{MIR}
\begin{pycode}
qmem.func @main() -> i1 {
    %qptr = qmem.qalloc : i32
    qmem.eprs %qptr
    qmem.hadamard %qptr
    %floats1 = qmem.recv_floats {remote = @Bob, length = 1 : i32} : tensor<1xf32>
    %t1 = tensor.extract %floats1[%zero] : tensor<1xf32>
    qmem.rot_x %qptr, %t1
    %m = qmem.measure %qptr : i1
    qmem.return %m : i1
}
\end{pycode}

\paragraph{MIR (functionized)}
\begin{pycode}
qmem.func @routine1() -> i32 {
    %qptr = qmem.qalloc : i32
    qmem.eprs %qptr
    qmem.return %qptr : i32
}

qmem.func @routine2(%qptr: i32) -> i32 {
    qmem.hadamard %qptr
    qmem.return %qptr : i32
}

qmem.func @routine3(%qptr: i32, %t1: f32) -> i1 {
    qmem.rot_x %qptr, %t1
    %m = qmem.measure %qptr : i1
}

qmem.func @main() -> i1 {
    %qptr = qmem.call @routine1() : () -> i32
    qmem.call @routine2(%qptr) : () -> i32
    %floats1 = qmem.recv_floats {remote = @Bob, length = 1 : i32} : tensor<1xf32>
    %t1 = tensor.extract %floats1[%zero] : tensor<1xf32>
    %m = call @routine3(%qptr1, %t1) : (i32, f32) -> i1
    qmem.return %m : i1
}
\end{pycode}

\paragraph{LIR}
\begin{pycode}
netqasm.request_routine @req1() -> i32 {
    %vqubit = netqasm.qalloc : i32
    netqasm.eprs %vqubit {remote = @Bob}
    return %vqubit : i32
}

netqasm.local_routine @subrt2(%vqubit: i32) {
    netqasm.hadamard %vqubit
    return
}

netqasm.local_routine @subrt3(%vqubit: i32, %num: i32, %denom: i32) {
    netqasm.rot_x %vqubit, %num, %denom
    %m = netqasm.measure %vqubit : i1
    return %m : i1
}

qoalahost.main_func @main() -> i1{
    %zero = arith.constant 0 : index
    %vqubit = qoalahost.call @req1() : () -> i32
    netqasm.call @subrt2(%vqubit, %num1, %denom1) : (i32, i32, i32) -> ()

    %floats1 = qoalahost.recv_floats {remote = @Bob, length = 1 : i32} : tensor<1xf32>
    %t1 = tensor.extract %floats1[%zero] : tensor<1xf32>
    %num1, %denom1 = func.call @conver_float_to_num_and_denom(%t1) : (f32) -> (i32, i32)

    %m = netqasm.call @subrt3(%vqubit, %num1, %denom1) : (i32, i32, i32) -> i1
    qoalahost.return %m : i1
}
\end{pycode}



\section{Evaluation}
\label{netqasm:sec:evaluation}
We evaluate two of the design choices that we made for \ac{NetQASM}:
    (1) exposing unit-modules to the \ac{CNPU} and
    (2) adding the possibility to use platform-specific flavors of instructions.
For both elements we study the difference in including them in \ac{NetQASM} versus not including them.
We do this by simulating a teleportation application and a blind quantum computation application.
These examples also showcase the ability of \ac{NetQASM} to express general quantum internet applications.

We have implemented a simulator, called SquidASM~\cite{git_squidasm}, that simulates a network in which end-nodes have the internal architecture as described in~\cref{netqasm:sec:abstract_model}, that is, with an \ac{CNPU} and a \ac{QNPU}.
The simulator internally uses NetSquid~\cite{netsquid}, which was made specifically for the simulation of quantum networks.
SquidASM executes programs written using the SDK (\cref{netqasm:sec:python-sdk}), including sending \ac{NetQASM} subroutines to the (simulated) \ac{QNPU}.
The code and data that were used to produce the results in this section can be found at~\cite{git_netqasm_paper_data}.

\begin{figure}[t]
    \centering
    \includegraphics[width=0.7\textwidth]{figures/netqasm/plots/paper_teleport_sweep_gate_noise.png}
    \caption{
        Average fidelity between the original state at the sender and
        the final state at the server, as a function of the depolarizing noise
        of the native two-qubit gate of the NV-platform, both for the case of
        performing step 6 after (\textbf{No unit modules}) and before
        (\textbf{Unit modules}) step 4 and 5. Execution time of the native
        two-qubit gate is set to 0.5 ms. The rest of the parameters used are
        listed in \cref{netqasm:sec:simulation}. Each point is the average over each of
        the six Pauli states as initial state, repeated 100 times.}
    \label{netqasm:fig:sweep_gate_noise}
\end{figure}

We evaluate the performance of \ac{NetQASM} by looking at the runtime quality of two applications, both consisting of two programs (one per node).
The first is a teleportation of a single qubit from a sender node to a receiver node.
We define the quality as the fidelity between the original qubit state at the sender and the final qubit state at the receiver.
The second application is a blind computation protocol which involves a client and a server.
The server effectively performs, blindly, a single-qubit computation on behalf of the client.
The protocol is a so-called \textit{verifiable blind quantum computation}~\cite{fitzsimons2017unconditionally}.
This means that some of the rounds of the protocols are \textit{trap rounds}.
We define the quality that we evaluate as the error rate of these trap rounds, since this indicates the blindness of the server.

We run these applications on SquidASM, where we simulate realistic quantum hardware.
Specifically, we simulate nodes based on nitrogen-vacancies (NV) in diamond, that can do heralded entanglement generation between each other.
The simulated hardware uses noise models that are also used in~\cite{coopmans2021netsquid}.
For more details, see~\cref{netqasm:sec:simulation}.

A note on how we chose what to evaluate and what not.
We listed several design considerations in ~\cref{netqasm:sec:design_considerations}.
We addressed these in our design decisions (\cref{netqasm:sec:design_decisions}).
For some of these, it is straightforward to see how they address a certain consideration, such as conditionals allowing for fast runtime feedback, and unit modules for allowing multitasking, as explained in~\cref{netqasm:sec:design_decisions}.
Also, fundamental requirements like remote-entanglement generation and shared memory have been addressed.
The remaining considerations, and our solutions, namely platform independence and memory virtualization using unit modules, are less trivial to evaluate just by looking at the design.
Therefore, we focus on the evaluation of these two design decisions.

In our evaluation, we focus specifically on the Nitrogen-Vacancy hardware for our nodes.
This has two reasons.
First, it is a promising hardware platform for quantum network nodes~\cite{Taminiau2014} which we know quite well since it is available in the lab, and we have even used \ac{NetQASM} in a simple test case running on nodes based on NV~\cite{pompili2021experimental}.
Second, the NV hardware is interesting since it has a restricted gate set and qubit topology, which is explained in more detail below.
Therefore, we expect that the use of unit modules and an NV-specific flavor makes a difference in terms of runtime quality.

\subsection{Unit modules}
\label{netqasm:sec:evaluation-unit-modules}
We ask ourselves the question whether it pays off to expose unit modules, that is, a qubit topology with gate- and entanglement information.
Specifically, we want to know if there are situations where knowing the unit module gives the \ac{CNPU} an opportunity to optimize the application in a way that is not possible when not knowing the unit module.
If so, we are interested in how much advantage this gives (in terms of the runtime quality defined above).

In the next section we show that there are indeed situations where knowledge of the unit module is advantageous.
It can be that the order in which \ac{NetQASM} instructions are issued in a subroutine is sub-optimal, since virtual qubit IDs may be mapped in such a way that the \ac{QNPU} has to move virtual qubits to different physical qubits in order to execute the instructions.
If the \ac{CNPU} layer does not know this mapping, it cannot know that the instructions are ordered sub-optimally.
With knowledge of the unit module, on the other hand, the \ac{CNPU} can optimize the order and the overall application performance is improved.

We consider a teleportation application where a \textit{sender} program teleports a single qubit to another \textit{receiver} program.
It is assumed that the underlying platform is based on nitrogen-vacancy centers in diamond (NV) and use well-established models for both the noise and operations supported on such platforms, see \cref{netqasm:sec:simulation}.
The sender program uses two qubits: one to create entanglement with the receiver (qubit E), and one to send (teleport) to the receiver (qubit T).
At some point, the sender measures both qubits, after which it sends the outcomes to the receiver so that it can do the relevant corrections on its received qubit.
We assume that the sender program is written in a higher-level language like, like in our SDK (\cref{netqasm:sec:python-sdk}), and in such a way that it first issues a measurement operation on qubit T, and then on E. However, due to the differences in characteristics of the physical qubits, as will be explained below, it is more efficient to first do the measurement on E, and then on T. Now we consider two scenarios, namely
\begin{itemize}
  \item \textbf{Unit-modules (UM)}.
        We assume that the sender program is written and executed on a software stack implementing \ac{NetQASM}, which means that the application's view of its quantum working memory is in the form of a unit module.
        This unit module contains information about the above-mentioned hardware restrictions, and therefore a compiler can take advantage of it by re-ordering the measurement operations while generating the \ac{NetQASM} subroutines to be sent to the \ac{QNPU}.
  \item \textbf{No unit-modules (NUM)}.
        In this case the software stack also implements \ac{NetQASM}, but without unit modules.
        Specifically, the application sees its quantum memory as just a number of uniform qubits.
        Therefore, a compiler for this application does not know about the hardware restrictions, and will construct \ac{NetQASM}-subroutines sent to the \ac{QNPU} without doing any optimization and leaves the order of the operations to be performed as they are specified in the high-level SDK.
\end{itemize}

Let's first go through the steps of the teleportation application:\newline
\begin{description}
  \item[\textit{sender}]:
        \begin{enumerate}
          \item Initialize qubit $q_t$ to be teleported in a Pauli state.
          \item Create entanglement with \textit{receiver} using qubit $q_s$.
          \item Perform CNOT gate with $q_t$ as control and $q_s$ as target.
          \item Perform Hadamard gate on $q_t$.
          \item Measure qubit $q_t$ and store outcome as $m_1$.
          \item Measure qubit $q_s$ and store outcome as $m_2$.
          \item Send $m_1$ and $m_2$ to \textit{receiver}.
        \end{enumerate}
  \item[\textit{receiver}]:
        \begin{enumerate}
          \item Receive entanglement with \textit{sender} using qubit $q_r$.
          \item Receive measurement outcomes from \textit{sender}.
          \item Apply correction operations on $q_r$ based on measurement outcomes.
        \end{enumerate}
\end{description}

\begin{figure}[t]
    \centering
    \includegraphics[width=0.7\textwidth]{figures/netqasm/plots/paper_teleport_sweep_gate_time.png}
    \caption{
        Average fidelity of the teleported state (left y-axis, solid lines) and total
        execution time of the teleportation application (right y-axis, dashed
        lines) as a function of the execution time of the native two-qubit gate
        of the NV-platform, both for the case of performing step 6 after
        (\textbf{No unit modules}) and before (\textbf{Unit modules}) step 4 and
        5. Dephasing parameter of the native two-qubit gate is set to 0.02. The
        rest of the parameters used are listed in \cref{netqasm:sec:simulation}. Each
        point is the average over each of the six Pauli state as initial state,
        repeated 100 times. In both figures, error bars are smaller than the drawn
        dots.}
  \label{netqasm:fig:sweep_gate_time}
\end{figure}

We will now consider the order of the steps of the \textit{sender}.
Firstly, we assume that the qubit to be teleported, $q_t$, is always created before the entanglement.
We motivate this assumption below. For this reason, steps 1--3 and 7 are fixed and cannot change.
However, we are free to do step 6 before step 4 and 5, since these single-qubit operations and measurements commute, as long as we are consistent with the outcomes $m_1$ and $m_2$.
Let's now consider what impact this decision of measuring $q_s$ before $q_t$ or not has on the quality of execution for a NV-platform.

One of the biggest restrictions on a NV-platform is the topology of the qubits.
In particular, the NV-platform has a single communication-qubit (electron) surrounded by some number of storage qubits (carbon spins), see for example \cref{netqasm:fig:topology}.
The single communication qubit is not only responsible for any remote entanglement generation but also for any two-qubit gate and is the only qubit that can be directly measured.
These restrictions require qubit states to be moved back and forth between the communication qubit and the storage qubits in order to free up the communication qubit, to create new entanglement or to measure another qubit.
Since the operation of moving a qubit state is relatively slow on this platform (up to a millisecond~\cite{Humphreys2018}) and adds noise to the qubits, it is important to try to minimize the number of moves needed.
For more details on the NV-platform, see for example~\cite{Bernien2014} or~\cite{dahlberg2019linklayer}.

In the steps of the \textit{sender} above, the communication qubit is first initialized to a Pauli state.
This state is then moved to a storage qubit to free up the communication qubit in order to create entanglement with the \textit{receiver}.
Then in step 5, $q_t$ should be measured, which is currently in the storage qubit.
This requires the qubit state to first be moved to the communication qubit.
However, at this point the communication qubit is occupied by the entangled pair and therefore first needs to be moved to a second storage qubit.
Qubit $q_t$ can then be moved to the communication qubit to be measured and then the same is done for $q_s$, requiring in total four move operations and three physical qubits.

We can now see that performing step 6 before 4 and 5 has the advantage that this qubit is already in the communication qubit and can be measured directly without moving it first.
Afterwards, $q_t$ can be moved to the communication qubit, which is cleared after the measurement, requiring in total only 2 move operations and only two physical qubits.
The decision of performing step 6 before 4 and 5 is highly dependent on the NV-platform and can only be made by a compiler that is aware about these restrictions.
The inclusion of unit-modules and qubit types in the \ac{NetQASM}-framework, which are exposed to the compiler at the \ac{CNPU}, allows for these optimization decision and can therefore improve the quality of execution.

For the two scenarios we consider, i.e. performing step 6 before 4 and 5 (\textbf{Unit modules (UM)}) or not (\textbf{No unit-modules (NUM)}), we check the average fidelity of the teleported state as a function of the gate noise (\cref{netqasm:fig:sweep_gate_noise}), as well as the average fidelity and execution time as a function of gate duration (\cref{netqasm:fig:sweep_gate_noise}), of the native two-qubit gate of the NV-platform. We see that performing step 6 before 4 and 5 improves both total execution time and average fidelity.
This can be explained by the fact that using unit modules allowed a compiler to produce \ac{NetQASM} code containing fewer two-qubit gates.
Therefore, an increase in two-qubit gate noise leads to a lower fidelity.
Also, an increase in two-qubit gate duration leads to higher execution time difference between the two scenarios.
Finally, \cref{netqasm:fig:sweep_gate_noise} shows that the two-qubit gate duration does not affect the final fidelity in this situation, but the difference between using unit modules versus not using them remains.






\subsection{Flavors}
\label{design_decisions_flavors}
While aiming to let \ac{NetQASM} be mostly platform-independent, we did also choose to allow platform-specific instructions, bundled in flavors.
The idea is that this allows for platform-specific optimization leading to better application performance.
Here we evaluate if flavors really impact potential performance, and if so how much.

We show that platform-specific optimization can indeed improve application performance, and that there are such optimizations that are not possible without flavors.
We see that it has impact mostly on the execution time, but not necessarily on outcome quality.

We consider the blind computation application depicted in \cref{netqasm:fig:bqc_app}, where both the client and server node implement the NV hardware.
Again we compare two scenarios, in this case:
\begin{itemize}
  \item \textbf{Vanilla}.
        We compile both the client's and server's application code to \ac{NetQASM} subroutines with the vanilla flavor.
        The \ac{QNPU}, controlling NV hardware which does not implement all vanilla gates natively, needs to translate the vanilla instructions on the go.
        We assume this translation is ad-hoc and does not do any optimizations like removing redundant gates.
  \item \textbf{NV}.
        The code is compiled to \ac{NetQASM} subroutines containing instructions in the NV flavor, and redundant gates are optimized away.
        The \ac{QNPU} can directly execute the instructions on the hardware.
\end{itemize}

We implemented this by writing two separate programs in the SDK, one for the client and one for the server.
The SDK automatically compiles the relevant parts of these programs into \ac{NetQASM} subroutines.
Classical communication (values $\delta_1$, $m_1$ and $\delta_2)$ is done purely between the two simulated \ac{CNPU}s, so these operations are not compiled to \ac{NetQASM} subroutines.
More details about the simulation can be found in \cref{netqasm:sec:simulation}.

The protocol is a verifiable blind quantum protocol~\cite{fitzsimons2017unconditionally}, which means that the circuit in~\cref{netqasm:fig:bqc_app} is run multiple times, namely once per round.
Some of these rounds are \textit{trap rounds} in which the client chooses a special set of input values.
Such a trap round can either succeed or fail, depending on the values returned by the server.
The fraction of trap rounds that fail is called the error rate.
The error rate should stay low in order for the computation to be blind.

We simulate the BQC application by running the client's and server's programs in SquidASM.
We look at the error rate of the trap round as a function of the two-qubit gate noise.
The result can be seen in~\cref{netqasm:fig:plot_bqc}.
It can be seen that using the NV flavor provides a better (lower) error rate than using the vanilla flavor.
This can be explained by noting that \ac{NetQASM} instructions in the vanilla flavor are mapped ad-hoc to native NV gates by the \ac{QNPU} at runtime, which leads to more two-qubit gates in total.


\begin{figure}[t]
  \centering
  \includegraphics[width=1.0\textwidth]{figures/netqasm/bqc_app.pdf}
  \caption{Circuit representation of the simulated BQC application. The client
    remotely prepares two qubits on the server, by twice creating an
    entangled pair with the server followed by a local measurement. The
    server locally entangles its two qubits (cphase gate). Then, the client
    and server use classical communication to further guide the server's
    quantum operations. The client computes $\delta_1 = \alpha - \theta_1 +
      p_1 \cdot \pi$ and sends this to the server. The server uses the
    received value to do a local rotation and later sends measurement
    outcome $m_1$ back to the client. The client then sends $\delta_2 =
      (-1)^{m_1} \cdot \beta - \theta_2 + p_2 \cdot \pi$ to the server.
    The qubit state $q$ is the result of this application.
  }
  \label{netqasm:fig:bqc_app}
\end{figure}


\begin{figure}[t]
  \centering
  \includegraphics[width=0.7\textwidth]{figures/netqasm/plots/bqc_sweep_gate_noise_trap.png}
  \caption{
    Average error rate of trap rounds for the circuit of~\cref{netqasm:fig:bqc_app}.
    Each point is the average over four combinations of $\theta_1$ and $\theta_2$,
    each used in 500 trap rounds. It can be seen that using the vanilla (platform-independent)
    \ac{NetQASM} flavor results in a worse (higher) error rate on average.}
  \label{netqasm:fig:plot_bqc}
\end{figure}

To gain some more insight into why using the NV flavor provides a lower error rate we also look at the fidelity of the two quantum states on the server before any local gates are applied.
As can be seen in~\cref{netqasm:fig:bqc_app}, the client remotely prepares two states on the server by twice creating entanglement and measuring its own half of the EPR pair.
In~\cref{netqasm:fig:plot_bqc_fidelity} we see that already during this remote state preparation phase the NV flavor outperforms the vanilla flavor in terms of the fidelity of the prepared states.

\begin{figure}[t]
  \centering
  \includegraphics[width=0.7\textwidth]{figures/netqasm/plots/bqc_sweep_gate_noise_epr_fidelity.png}
  \caption{ Fidelity of the two remotely prepared states on the server in
        the BQC application. To remotely prepare a state, the client and server
        first create an EPR pair, and the client then measures its half in a
        specific basis while the server keeps its half stored in the
        communication qubit. This first prepared state is then moved to a memory
        qubit to free up the communication qubit for preparing the second state.
        This move operation has a negative effect on the fidelity of the first
        prepared state. Since the fidelity of the second prepared state only
        depends on the link entanglement generation, there is no difference
        between using vanilla or NV instructions. The values are from the same
        simulation experiment as for ~\cref{netqasm:fig:plot_bqc}. Error bars are
        negligible.}
  \label{netqasm:fig:plot_bqc_fidelity}
\end{figure}

\subsection{Relation to other results}
We note that a similar question of how many physical details to expose from lower-level layers (in our case the \ac{QNPU}) to higher-level layers (in our case the \ac{CNPU}) has also been evaluated in~\cite{murali2019fullstack}.
Their conclusion is that exposing and leveraging some of these details can indeed improve certain program success metrics.
That result agrees with that of ours, which shows that program execution quality can improve by exposing and leveraging unit modules and platform-specific \ac{NetQASM} flavors.


\section{Conclusion}
\label{qoala:sec:conclusion}
Qoala is the first architecture for executing quantum applications that addresses the need for scheduling and compiling hybrid classical-quantum programs for a quantum internet.
This allows Qoala to ensure successful execution of quantum programs even in the presence of limited quantum memory lifetimes, and opens the door for a compile time optimization of the hybrid classical-quantum program.
By building on an existing quantum network stack~\cite{dahlberg2019link, pompili2022experimental} and the implementation of QNodeOS on quantum hardware~\cite{pompili2022experimental, carlothesis} we pave the way for the real-world implementation of Qoala in a platform-independent way on diverse hardware platforms including NV centers in diamond~\cite{pompili2021realization, pompili2022experimental} and trapped ions~\cite{krutyanskiy2023entanglement,krutyanskiy2023telecom}.
Such an implementation may require, however, a new classical control hardware as opposed to~\cite{pompili2022experimental, carlothesis}, e.g. by placing CPS and QPS on a single board with access to an on-chip shared memory. 

Our work opens the door for further computer science research in executing quantum internet applications:
\textit{Advanced scheduling algorithms:}
More sophisticated scheduling strategies may lead to higher success probabilities and lower makespan when concurrently executing multiple program instances, where inspiration may come from~\cite{topcuoglu2002performance, baruah2011scheduling, andersson2006multiprocessor, polychronopoulos1991hierarchical}. 
In the quantum domain, missing the deadline will result in a degradation of the success probability as a function of the time by which the deadline was exceeded.
This suggest the use of time-utility functions (TUF, see e.g.~\cite{jensen1993timeliness, li2004utility}) to inform scheduling decisions, where it is an open question how such TUF could even be defined in the quantum domain.
Our work also raises the question on what fundamental tradeoffs between the classical (makespan) and quantum (success probability) performance metrics are at all possible.
\textit{Compiler design:}
Qoala's program format now allows for a compiler design that takes into account the hybrid and networked nature of programs.
It is an open question to design compilers enabling effective code optimization and translation of different types of high-level code into executables.
\textit{Capability negotiation:}
We assumed that the compiler provides advice that the nodes use in a capability negotiation and demand registration (\cref{qoala:sec:program_instantiation}).
It is an open question how to best compute such advice, and find efficient protocols for negotiating capabilities and register demand.
\textit{Network schedule:}
As expected, our evaluation shows that application performance depends on the network schedule, where we emphasize that ensuring network service is out of scope for Qoala as an environment for executing applications.
This highlights a need for understanding the quality of service a quantum network should provide, as well as to design good network scheduling algorithms to satisfy them, in order to achieve good application performance.


\section{Data availability}
The implementation of Qoala as a simulator can be found online~\cite{qoala2023simulator}.
The code and data supporting the evaluation can be found at~\cite{qoala-evaluation-data}.

% \addtocontents{toc}{\protect\setcounter{tocdepth}{1}}
\input{chapters/main/qoala/appendix/program_structure}
\input{chapters/main/qoala/appendix/runtime_environment}
\input{chapters/main/qoala/appendix/scheduler}
\input{chapters/main/qoala/appendix/simulator}
\input{chapters/main/qoala/appendix/evaluation_details}
% \addtocontents{toc}{\protect\setcounter{tocdepth}{2}}
\fi


\begin{xstretch}
\printbibliography[heading=subbibintoc,title={References},notcategory=noprint]
\end{xstretch}