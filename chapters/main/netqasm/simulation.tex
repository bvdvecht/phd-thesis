\section{Simulation details}
\label{netqasm:sec:simulation}

In this section we detail how simulations in \cref{netqasm:sec:evaluation} were
performed and what models and parameters were used. All simulations used the
\ac{NetQASM} SDK~\cite{git_netqasm}, using
NetSquid~\cite{netsquid,coopmans2021netsquid} as the underlying simulator. All
code used in these simulations can also be found at~\cite{git_squidasm}.


\subsection{Noise model}
In both the teleportation and the blind quantum computing scenario we used the
same model for nitrogen-vacancy centres in diamonds as was used
in~\cite{dahlberg2019linklayer} and~\cite{coopmans2021netsquid}. All gates
specified by the application in the SDK were translated to NV-specific gates,
see \cref{tab:gates}, using a simple compiler without any optimization. The
parameters used in the model from~\cite{dahlberg2019linklayer} are listed in
\cref{tab:gates,tab:noise}, together with an explanation and a reference.
\nq{ec_controlled_dir_xy} are the native two-qubit gates of the NV-platform,
ideally performing one of the unitary operations
\begin{align}
  U_\mathrm{ec_x}(\alpha) & = \begin{pmatrix}R_x(\alpha) & 0 \\ 0 & R_x(-\alpha) \end{pmatrix} \label{eq:crot_x} \\
  U_\mathrm{ec_y}(\alpha) & = \begin{pmatrix}R_y(\alpha) & 0 \\ 0 & R_y(-\alpha) \end{pmatrix} \label{eq:crot_y} \\
\end{align}
where $R_x(\alpha)$ and $R_y(\alpha)$ are the rotation matrices around $X$ and
$Y$, respectively. When sweeping the duration and noise of this two-qubit gate
the same value is also used for the \nq{carbon_xy_rot} ($X$- and $Y$-rotations
on the carbon) on the storage qubits, since these are also effectively done with
a similar operation also involving the communication qubit (electron). All noise
indicated by a fidelity in \cref{tab:noise} are applied as depolarising noise by
applying the perfect operation, producing the state $\rho_\mathrm{ideal}$, and
mapping this to
\begin{equation}\label{eq:depolarising}
  \rho_\mathrm{noisy}=(1-p)\rho_\mathrm{ideal} + \frac{p}{3}X\rho_\mathrm{ideal}X + \frac{p}{3}Y\rho_\mathrm{ideal}Y + \frac{p}{3}Z\rho_\mathrm{ideal}Z
\end{equation}
where $X$, $Y$ and $Z$ are the Pauli operators in
\cref{neqtasm:eq:pauli_x,netqasm:eq:pauli_y,netqasm:eq:pauli_z}, $p=\frac{4}{3}(1 - F)$, with $F$ being
the value specific in \cref{tab:noise}. Decoherence noise is specific as $T_1$
(energy/thermal relaxation time) and $T_2$ (dephasing time)~\cite{Nielsen2010}.

\begin{table}
  \centering
  \begin{tabular}{|c|c|c|}
    \hline
    Gate                      & Durations (ns) & Explanation                                                \\
    \hline\hline
    \texttt{electron\_init}        & 2e3            & Initialize a comm. qubit (electron) to $|0\rangle$ \\
    \texttt{electron\_rot}         & 5              & single-qubit rotation on comm. qubit (electron)    \\
    \texttt{measure}              & 3.7e3          & Measure communication qubit (electron)                     \\
    \texttt{carbon\_init}          & 3.1e5          & Initialize a storage qubit (carbon) to $|0\rangle$         \\
    \texttt{carbon\_xy\_rot}        & $t$            & $X$/$Y$-rotation on storage qubit (carbon)                 \\
    \texttt{carbon\_z\_rot}         & 5              & $Z$-rotation on storage qubit (carbon)                     \\
    \texttt{ec\_controlled\_dir\_xy} & $t$            & Native two-qubit gates, (\cref{eq:crot_x,eq:crot_y})     \\
    \hline
  \end{tabular}
  \caption{
    Gate durations for scenario \textbf{B} of \cref{netqasm:sec:evaluation}.
    $t$ is the value being swept in \cref{netqasm:fig:sweep_gate_time}.
    All values are from~\cite{dahlberg2019linklayer}.}
  \label{tab:gates}
\end{table}

\begin{table}
  \centering
  \begin{tabular}{||c|c|c|c||}
    \hline
    Parameter                 & Value        & Explanation                                                 \\
    \hline\hline
    \texttt{electron\_T1}          & 1 hour       & $T_1$ of communication qubit (electron)                     \\
    \texttt{electron\_T2}          & 1.46 seconds & $T_2$ of communication qubit (electron)                     \\
    \texttt{electron\_init}        & 0.99)        & Fidelity to initialize comm. qubit (electron)       \\
    \texttt{electron\_rot}         & 1.0          & Fidelity for $Z$-rotation on comm. qubit (electron) \\
    \texttt{carbon\_T1}            & 10 hours     & $T_1$ of storage qubit (carbon)                             \\
    \texttt{carbon\_T2}            & 1 second     & $T_2$ of storage qubit (carbon)                             \\
    \texttt{carbon\_init}          & 0.997        & Fidelity to initialize storage qubit (carbon)               \\
    \texttt{carbon\_z\_rot}         & 0.999        & Fidelity for $Z$-rotation on storage qubit (carbon)         \\
    \texttt{carbon\_xy\_rot}        & $f$          & Fidelity for $X$/$Y$-rotation on storage qubit (carbon)     \\
    \texttt{ec\_controlled\_dir\_xy} & $f$          & Fidelity for native two-qubit gate                          \\
    \texttt{prob\_error\_meas\_0}    & 0.05         & Probability of flipped measurement outcome for $|0\rangle$  \\
    \texttt{prob\_error\_meas\_1}    & 0.005        & Probability of flipped measurement outcome for $|1\rangle$  \\
    \texttt{link\_fidelity}        & 0.9          & Fidelity of generated entangled pair.                       \\
    \hline
  \end{tabular}
  \caption{
    Noise parameters for used in the simulations of \cref{netqasm:sec:evaluation}.
    $f$ is the value being swept in \cref{netqasm:fig:sweep_gate_noise} and \cref{netqasm:fig:plot_bqc}.
    All fidelities are realized by a applying depolarising noise as in \cref{eq:depolarising}.
    All values are from~\cite{coopmans2021netsquid}, except \nq{link_fidelity} which is set to relatively high value to avoid this being the major noise-contribution and preventing any conclusions to be made.
  }
  \label{tab:noise}
\end{table}

\subsection{BQC application and flavors}
In \cref{design_decisions_flavors} we simulated the blind quantum computation
(BQC) application from \cref{netqasm:fig:bqc_app}. The code for this is available at
\cite{git_squidasm}.

In the scenario when the application code was compiled to subroutines with the
vanilla flavor, the \ac{QNPU} had to map the vanilla instructions to NV-native
operations on the fly. We used the gate mappings listed below. For
convenience we use \nq{PI} and \nq{PI_OVER_2} for $\pi$ and $\frac{\pi}{2}$
respectively.

A \nq{h} (Hadamard) vanilla instruction was mapped to the following NV instruction sequence:
\begin{nqcode}
  rot_y PI_OVER_2
  rot_x PI\end{nqcode}

A \nq{cnot C S} vanilla instruction between a communication qubit (C) and a storage
qubit (S) (as specified in the unit module) was mapped to the following NV
instruction sequence:
\begin{nqcode}
  cx_dir C S PI_OVER_2
  rot_z C -PI_OVER_2
  rot_x S -PI_OVER_2\end{nqcode}

A \nq{cnot S C} vanilla instruction between a store qubit (S) and a communication
qubit (C) (as specified in the unit module) was mapped to the following NV
instruction sequence:
\begin{nqcode}
  rot_y C PI_OVER_2
  rot_x C PI
  rot_y S PI_OVER_2
  cx_dir C S PI_OVER_2
  rot_z C -PI_OVER_2
  rot_x S -PI_OVER_2
  rot_y S PI_OVER_2
  rot_y C PI_OVER_2
  rot_x C PI\end{nqcode}

A \nq{cphase C S} vanilla instruction between a communication qubit (C) and a storage
qubit (S) (as specified in the unit module) was mapped to the following NV
instruction sequence:
\begin{nqcode}
  rot_y S PI_OVER_2
  cx_dir C S PI_OVER_2
  rot_z C -PI_OVER_2
  rot_x S -PI_OVER_2
  rot_y S -PI_OVER_2\end{nqcode}