\section{Python SDK}
\label{netqasm:sec:python-sdk}
We implemented \ac{NetQASM} by developing a Software Development Kit (SDK) in Python.
This SDK allows a programmer to write quantum network programs as Python code, including the quantum parts.
These parts are automatically translated to NetQASM subroutines.
The SDK contains a simulator that simulates a quantum network containing end-nodes, each with a \ac{QNPU}.
The SDK can execute programs by executing their classical parts directly and executing the quantum parts as \ac{NetQASM} subroutines on the simulated \ac{QNPU}.
By executing multiple programs at the same time, on the same simulated network, a whole multi-partite application can be simulated.
In~\cref{netqasm:sec:evaluation} we use this SDK to evaluate some of the design decisions of \ac{NetQASM}.

We refer to the docs at~\cite{git_netqasm} for the latest version of the SDK.
Below, we give an example of an application written in the SDK to give an idea of how development in the SDK looks like.
In \cref{app:examples_sdk} we provide a few more examples of applications in the SDK and their corresponding \ac{NetQASM} subroutines.

All code can be found at~\cite{git_netqasm} and~\cite{git_squidasm}, including:
    (1) Tools for serializing (de-serializing) to (from) both human-readable text form and binary encoding,
        % \item A parser for the human-readable form of \ac{NetQASM}.
        % \item A parser for binary encoding of \ac{NetQASM}.
        % \item An assembler of human-readable form to binary.
    (2) the \ac{NetQASM} SDK, together with compilers (no optimization yet),
    (3) support for running applications written in the SDK on the simulators NetSquid~\cite{netsquid,coopmans2021netsquid} and SimulaQron~\cite{dahlberg2018simulaqron}, and
    (4) implemented applications in \ac{NetQASM}, including: anonymous transmission~\cite{Christandl2005anonymous}, BB84~\cite{bb84}, blind quantum computing~\cite{broadbent2009universal,fitzsimons2017unconditionally}, CHSH game~\cite{Kaniewski2016}, performing a distributed CNOT~\cite{denchev2008distributed}, magic square game~\cite{brassard1999magicsquare}, teleportation~\cite{bennett1993teleporting}.

\paragraph{SDK}
The SDK of \ac{NetQASM} uses a similar framework to the SDK used by the predecessor \ac{CQC}~\cite{git_cqc}.
Any program on a node starts by setting up a \texttt{NetQASMConnection} to the \ac{QNPU}-implementation in the \emph{backend}.
The \texttt{NetQASMConnection} encapsulates all communication that the \ac{CNPU} does with the \ac{QNPU}.
More information about supported backends can be found below in \cref{netqasm:sec:backends}.
Using the \texttt{NetQASMConnection} one can for example construct a \texttt{Qubit} object.
The \texttt{Qubit} object has methods for performing quantum gates and measurements.
When these methods are called, corresponding \ac{NetQASM} instructions are included in the current subroutine being constructed.
One marks the end of a subroutine, and the start of another, either by explicitly calling \texttt{flush} on the \texttt{NetQASMConnection} or by ending the scope of the \nq{with NetQASMConnection ...} context.

The following Python code shows a basic application written in the \ac{NetQASM} SDK.
The application will be compiled into a single subroutine executed on the \ac{QNPU}, which creates a qubit, performs a Hadamard operation, measures the qubit and returns the result to the application layer.
\subsubsection{Hello world/Hello Hadamard}
Functionally the same as the \ac{NetQASM}-subroutine~\ref{netqasm:sec:example_nq_hello_world}.
\begin{pycode}
  # Setup connection to backend
  # as the node Alice
  with NetQASMConnection("Alice") as alice:
    # Create a qubit
    q = Qubit(alice)
    # Perform a Hadamard on the qubit
    q.H()
    # Measure the qubit
    m = q.measure()
    # The end of the context also marks
    # the end of the subroutine
    # automatically but can also be done
    # explicitly using `alice.flush()`
\end{pycode}

The following \ac{NetQASM} subroutine is the result of translating the above Python code to \ac{NetQASM} of the vanilla (platform-independent) flavor.

\subsubsection{Hello world/Hello Hadamard}\label{netqasm:sec:example_nq_hello_world}
A simple subroutine which creates a qubit, performs a Hadamard, measures the qubit and returns the measurement outcome to the host.
\begin{nqcode}
  # NETQASM 1.0
  # APPID 0
  // Set the virtual qubit ID to use
  set Q0 0

  // Allocate and initialize a qubit
  qalloc Q0
  init Q0

  // Perform a Hadamard gate
  h Q0

  // Measure the qubit
  meas Q0 M0

  // Return the outcome
  ret_reg M0
\end{nqcode}

\subsubsection{Backends}
\label{netqasm:sec:backends}
As mentioned above, the \texttt{NetQASMConnection} in the SDK is responsible for communicating with the implemented \ac{QNPU} in the \emph{backend}.
The \emph{backend} can either be a simulator or an actual \ac{QNPU} using real quantum hardware.
Currently supported backends are the simulators SquidASM~\cite{git_squidasm} (using NetSquid~\cite{netsquid, coopmans2021netsquid}) and SimulaQron~\cite{dahlberg2018simulaqron}.
A physical implementation of \ac{QNPU} running on quantum hardware is being worked on at the time of writing.
Using the SDK provided at~\cite{git_netqasm}, one can for example simulate a set of program files for the nodes of a quantum network on NetSquid using a density matrix formalism with the command:
\begin{nqcode}
  netqasm simulate --simulator=netsquid --formalism=dm
\end{nqcode}
For more details see the docs at~\cite{git_netqasm}.