\chapter*{Summary}
\addcontentsline{toc}{chapter}{Summary}
\setheader{Summary}


Quantum networks consisting of interconnected quantum processors enable applications that are impossible on non-quantum (or \emph{classical}) computers and networks such as quantum key distribution and secure quantum computing in the cloud.
Scaling up of quantum networks is expected to result in a global \textit{quantum internet}, in which, similar to the classical internet, independent quantum devices can participate and communicate with each other.
For classical computers and networks, realization of new applications relies on programming languages to write code, compilers to translate that code into machine-readable instructions, and runtime environments (such as operating systems) to execute applications.
To enable similar development for quantum network (or internet) applications, and indeed to realize useful applications on potentially large-scale quantum networks, such frameworks are needed as well.
However, they do not exist.

To reach this goal, we present in this thesis the first system and software architectures that enable programming and execution of arbitrary quantum network applications in a hardware-agnostic way while optimizing runtime performance.

We start by introducing NetQASM --- a programming representation for quantum network applications.
This representation includes a new low-level instruction set architecture tailored to quantum network applications, but it also contains a high-level software development kit, enabling developers to express their application logic.
We purposefully make NetQASM hardware-independent and extendible.

We then build a detailed full-stack system architecture for executing arbitrary applications on quantum network nodes: QNodeOS.
We implement this architecture on a setup with two real physical quantum network nodes, and show that our architecture can successfully execute quantum network applications.
We report on the performance of our architecture by looking at application throughput and success probability.

Based on what we learned from our first implementation and evaluation, we propose an improved architecture --- Qoala --- addressing the compilation and scheduling challenges that we found, by allowing hybrid classical-quantum compilation and scheduling.
We show how this architecture enables strategies to achieve better application performance.
Finally, we discuss an architecture for Qoala program compilers.

