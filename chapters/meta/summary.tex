\chapter*{Summary}
\addcontentsline{toc}{chapter}{Summary}
\setheader{Summary}

Quantum networks consist of interconnected quantum computers, similar to how `normal' (\emph{classical}) computers and devices are connected together in networks like the internet.
By using quantum mechanical phenomena like superposition and entanglement, quantum networks enable applications that are impossible with classical computers and networks, like extra secure communication and doing quantum computations in the cloud.
So far, only simple `proof of concepts' have been demonstrated on small-scale quantum networks that were optimized specifically for those demonstrations.
To improve usage of quantum networks, and accelerate adoption, it is necessary to use (software) abstractions and tools that allow for flexibly programming and executing new applications on the \emph{nodes}---the quantum devices and computers.
Here it is important that (software) developers are able to write \emph{computer programs} without requiring knowledge of the underlying (quantum) mechanisms and that nodes possess an \emph{operating system} that is able to execute these programs.
Such abstractions and tools do however not yet exist for quantum networks.

In this thesis, we therefore present new system and software architectures that enable, for the first time, programming and execution of arbitrary quantum network programs.
This involves a number of challenges.
We would like to stay independent of specific quantum hardware; we need to deal with the fact that quantum memory quality decreases (very quickly) over time; and a question is how exactly we should represent and execute the mix of classical and quantum operations.

We present a series of architectures that build on top of each other.
First we introduce NetQASM --- an \emph{instruction set} for quantum network programs that contains instructions for making entanglement with other quantum devices in the network.
Moreover we present a \emph{software development kit} --- a toolbox for software developers to program quantum network applications without having to deal with underlying quantum hardware.

Then, we present an \emph{operating system} for nodes --- QNodeOS --- that is able to execute arbitrary programs that have been programmed using NetQASM.
We implement QNodeOS and test it successfully on a real quantum network in the lab consisting of two small quantum computers.
Furthermore we show that QNodeOS is able to \emph{multitask} which leads to more efficient usage of the hardware.

We proceed by investigating how to improve the quality of applications by focusing on both \emph{scheduling} and \emph{compilation} of programs.
This leads to a new design --- Qoala, building on top of QNodeOS --- in which a scheduler manages both classical and quantum tasks, and where a compiler can optimize program code and translate it to executable instructions.

