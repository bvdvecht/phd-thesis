\chapter*{Summary}
\addcontentsline{toc}{chapter}{Summary}
\setheader{Summary}

\todo{}

% Computers and computer networks enable a wide variety of applications.
% \emph{Quantum} computers and quantum networks present opportunities for applications that are impossible on non-quantum (or \emph{classical}) computers and networks.
% Scaling up of quantum networks is expected to result in a global \textit{quantum internet}, in which, similar to the classical internet, arbitrary quantum devices can participate and communicate with each other.
% Development of new applications relies on programming languages to write the code, compilers to translate that code into machine-readable instructions, and runtime environments (such as operating systems) to execute the application.

% In this thesis, we address the gap that exists for quantum networks: namely that there is no programming and execution framework for quantum network (or internet) applications.
% The main goal is to \emph{enable programming and execution of arbitrary quantum network applications in a hardware-agnostic way while optimizing runtime performance}.

% To reach this goal, we present new architectures, tools, and paradigms.
% We present a programming representation for quantum network applications.
% This representation includes a new low-level instruction set architecture tailored to quantum network applications,
% but it also contains a high-level software development kit, enabling developers to express their application logic.
% We purposefully make NetQASM hardware-independent and extendible.

% We then build a detailed full-stack system architecture for executing arbitrary applications on quantum network nodes.
% We implement this architecture on a setup with two real physical quantum network nodes, and show that our architecture can successfully execute quantum network applications.
% We report on the performance of our architecture by looking at application throughput and success probability.

% Based on what we learned from our first implementation and evaluation, we propose an improved architecture, addressing the compilation and scheduling challenges that we found, by allowing hybrid classical-quantum compilation and scheduling.
% We show how this architecture enables strategies to achieve better application performance.
% Finally we discuss how compilers may be written.
