\chapter*{Acknowledgments}
\addcontentsline{toc}{chapter}{Acknowledgments}
\setheader{Acknowledgments}

This thesis is the result of more than four years of research and development, and would not have been possible without the help of many people.
During my PhD at QuTech, I have had the privilege of being surrounded by a diverse group of smart, interesting, and hard-working people.
All people whom I've interacted with have shaped, either directly or more indirectly, the way I've worked through, sometimes battled through, my PhD.

First of all, many thanks to my promotor \textbf{Stephanie Wehner}.
It was an honor to have been a part of working towards your vision for a quantum internet.
I admire your ability to keep up many, many balls (although I'm not jealous of your busy calendar)
and I enjoyed the balance you gave me between freedom and responsibility as well as giving very valuable feedback and guidance when needed.
Although I've at times felt more an engineer than a scientist, you have always been supportive and motivated me about continuing the PhD, even when I had serious doubts.
I look forward to see where your ambition for the quantum internet leads to, and I'm glad I'll remain a part of the wider endeavour.

I would also like to thank my committee members for taking the time to read through my thesis and subject me to critical questions during my defense:
\textbf{Arie van Deursen}, acting as second promotor;
\textbf{Rodney van Meter}, joining online all the way from Japan;
\textbf{Michele Amoretti}, who I've had the pleasure of interacting with in various QIA- and compiler-discussion-related settings;
\textbf{Tim Coopmans}, who was a fellow PhD student at QuTech when I joined and I have the honor to have him in this new role;
\textbf{Rob Kooij}, who was so kind to agree to be in the committee after 9 other TU Delft full professors happened not to be available;
and \textbf{Koen Langendoen}, for being available as a reserve member.

Thanks to \textbf{Soham Chakraborty} and \textbf{Johannes Borregaard} for being in my Go/No Go commitee.

Of course many thanks to my wonderful paranymphs, \textbf{Carlo} and \textbf{Francisco}.
Carlo, it has been a pleasure sharing an office for a long time. Your genuine interest in other people is admirable.
Francisco, I envy your sharp wits! Looking forward to all the use cases you'll cook up for the quantum internet.

Special thanks to \textbf{Axel} who, in the chaotic situation of the initial lockdown and a supervisor on maternity leave, managed to on-board me and inspire me in my first months of the PhD.
It has been an honor to use your work as a starting point for my PhD.
% The first two years of my PhD were mostly during (partial) lockdowns and other restrictions and therefore mostly remote.

Other people in my group that I've had the pleasure of interacting with during the beginning of my PhD include
\textbf{Aram}, \textbf{Matt}, \textbf{Guus}, and \textbf{David}.

\textbf{\'Alvaro}, \textbf{Bethany}, \textbf{Scarlett}, you're all nearing the end of your PhD, and you can all be proud of what you've achieved!
\textbf{Janice}, \textbf{Jeroen}, \textbf{Kaushik}, \textbf{Luca}, \textbf{Margrete}, \textbf{Soubhadra}, \textbf{Sounak}, \textbf{Thomas}, \textbf{Tzula}, thanks for being awesome group members.
\textbf{Felix}, \textbf{Sacha}, \textbf{Sam}, you've just started but I'm already seeing great things from you!
Sam and Sacha: I'm looking forward to the cool things you're going to do with Qoala!
\textbf{Diego}, your enthousiasm for code development has been a great addition to our group.

Not only my own group but also other people at QuTech have inspired me.
\textbf{Eric}, \textbf{Fenglei}, \textbf{Hemant}, \textbf{Kenneth}, \textbf{S\'ebastian}, \textbf{Siddanth}, thanks for being either great office neighbours, good chess opponents, or just nice colleagues!

Special thanks to \textbf{Mariagrazia}, I really enjoyed doing the QNodeOS experiments together and we can be very proud about the result!
Also thanks to \textbf{Przemek}, \textbf{Tracy} and all other co-authors of the QNodeOS experiments.

I also feel very privileged for all the trips abroad that I was allowed to make, for conferences and research visits.
From Bad Honnef to Barcelona, from Paris to Austin, and from Seattle to Bali, those trips always left me inspired and with renewed energy.
Special thanks to \textbf{Carmina} and \textbf{Anabel} for hosting me in Valencia.
\textbf{Arithra}, \textbf{Luise}, \textbf{Matt}, \textbf{Medina}, \textbf{Pablo}, and \textbf{Sebastian}, I had a lot of fun with you in Seattle.

Thanks to the master students and interns that I supervised - you all helped me in my own work.
\textbf{Bob}, your project really helped kickstart the Qoala project.
% Emir?
\textbf{Atak}, it was a pleasure working with you and you contributed greatly to the Qoala simulator!
\textbf{Hana}, thanks for being such a great master student that I actually didn't have to do much myself.
I'm looking forward to working more with you in the Stack Team!

Speaking about the Stack Team,
\textbf{Guilherme}, \textbf{Ingmar} and \textbf{Thom}, \textbf{Wojciech}, you all have been crucial in the development of QNodeOS.
I'm looking forward to (keep) working with some of you in the future!

Thanks also to all software engineers at QuTech that have contributed to projects I've been involved in.
\textbf{Michal}, thanks for taking over the SquidASM development! It's in very capable hands.
\textbf{Mark}, \textbf{Fer}, \textbf{Olaf}, \textbf{Ivo}, \textbf{Animesh}, \textbf{Bob}, \textbf{Giuliano} and other software engineers that have helped with QNE.
\textbf{Paul}, \textbf{Ravi B}, thanks for hanging out in Leuven.
\textbf{Ravi V}, I liked our more software related discussions, and I'm happy that you found a role in QuTech that really fits you.


A big shoutout to the \textbf{QuTech band (T2 stars, formerly Q2)}.
Thanks for the great rehearsals every Wednesday and all the awesome performances that we did.
Over the years there have been too many members to list them all here, but one thing is for sure: you're all insanely talented.

This PhD would also not have been possible without great ways to spend time outside of QuTech.
Special thanks to my fellow \textbf{Tiramisu} band members, \textbf{David}, \textbf{Floris} and \textbf{Pim}.
We'll make that album one day!
Shoutout to the \textbf{DSBA-USSR Badminton club} for keeping me fit and helping me meet new friends.
I'd also like to mention \textbf{Boris}, \textbf{Dirk}, \textbf{Jonas}, \textbf{Marianne}, \textbf{Mark} and \textbf{Shuham} for intellectual (but also just fun) philosophical discussions in our bookclub.
It really helped to talk about non-computer-science or physics things.

Natuurlijk veel dank aan mijn familie, \textbf{Papa}, \textbf{mama} en \textbf{Peter}, op wie ik altijd kan terugvallen.
En als laatste \textbf{Anne}, mijn grote steun, met jouw liefde kan ik alles aan.



% Gustavo, Sander, Luc, Timo, Laurens, Hanifa, Maia, Desa, Pablo, "Q2 band" (or should I say already T2* ?)

% Tiramisu
% Boekenclub
% Arminius
% Badminton


% \begin{flushright}
% {\itshape
% Bart \\
% Delft, Maart 2025.
% }
% \end{flushright}
