\chapter*{Samenvatting}
\addcontentsline{toc}{chapter}{Samenvatting}
\setheader{Samenvatting}

{\selectlanguage{dutch}

Kwantumnetwerken bestaan uit kwantumcomputers die met elkaar verbonden zijn, net zoals `normale' (\emph{klassieke}) computers en apparaten verbonden zijn in netwerken zoals het internet.
Door gebruik te maken van kwantummechanische verschijnselen als superpositie en verstrengeling, bieden kwantumnetwerken toepassingen die onmogelijk zijn met klassieke computers en netwerken, zoals extra beveiligde communicatie en het doen van kwantumberekeningen in de cloud.
Tot nu toe zijn er slechts simpele `proof of concepts' gedemonstreerd op kleinschalige kwantumnetwerken die speciaal voor die demonstratie waren geoptimaliseerd.
Om gebruik van kwantumnetwerken te bevorderen, en daarmee adoptie te versnellen, is het nodig om (software)abstracties en tools te gebruiken waardoor het mogelijk is om verschillende, nieuwe toepassingen flexibel te programmeren en uit te voeren op de \emph{nodes}---de kwantumapparaten- en computers.
Daarbij is het belangrijk dat (software)ontwikkelaars \emph{computerprogramma's} kunnen schrijven zonder verstand te hoeven hebben van de onderliggende (kwantum)mechanismes, en dat nodes een \emph{besturingssysteem} hebben dat in staat is zulke programma's uit te voeren.
Dit soort abstracties en tools bestaan op het moment echter nog niet voor kwantumnetwerken.

In dit proefschrift presenteren wij daarom nieuwe systeem- en softwarearchitecturen die het voor het eerst mogelijk maken om willekeurige kwantumnetwerkprogramma's te programmeren en uit te voeren.
Hierbij komt een aantal uitdagingen kijken.
Zo willen we graag onafhankelijk blijven van specifieke kwantumhardware, moeten we rekening houden met het feit dat de kwaliteit van kwantumgeheugen met de tijd (erg snel) afneemt, en is het de vraag hoe we precies de mix van klassieke- en kwantumoperaties moeten representeren en uitvoeren.

We presenteren een reeks architecturen die op elkaar voortbouwen.
Eerst introduceren we NetQASM --- een \emph{instructieset} voor kwantumnetwerkprogramma's dat instructies bevat voor het maken van verstrengeling met andere kwantumapparaten in het netwerk.
Ook presenteren we een \emph{software development kit} --- een gereedschapskist voor softwareontwikkelaars om kwantumnetwerktoepassingen te programmeren zonder rekening te hoeven houden met de onderliggende kwantumhardware.

Vervolgens presenteren we een \emph{besturingssysteem} voor nodes --- QNodeOS --- dat in staat is om willekeurige programma's uit te voeren die zijn geprogrammeerd met NetQASM.
We implementeren QNodeOS en testen het met succes op een echt kwantumnetwerk in het lab bestaande uit twee kleine kwantumcomputers.
We laten ook zien dat QNodeOS kan \emph{multitasken} waardoor we efficiënter gebruik maken van de hardware.

Daarna onderzoeken we hoe we de kwaliteit van toepassingen kunnen verbeteren door te focussen op zowel \emph{scheduling} (taakplanning) als \emph{compilatie} (vertaling) van programma's.
Dit leidt tot een nieuw ontwerp --- Qoala, dat voortbouwt op QNodeOS --- waarin een \emph{scheduler} het overzicht houdt over zowel klassieke- als kwantumtaken en waarin een \emph{compiler} programmeercode kan optimaliseren en vertalen naar uitvoerbare instructies.


}
